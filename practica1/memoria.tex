\documentclass[]{article}
\usepackage{graphicx}
\usepackage[spanish]{babel}
\usepackage[a4paper, top=2.5cm, bottom=2.5cm, left=3cm, right=3cm]{geometry}
\usepackage[hidelinks]{hyperref}
\usepackage[T1]{fontenc}
\usepackage{listings}
\usepackage{xcolor}
\usepackage{float}

\definecolor{miverde}{rgb}{0,0.6,0}

% style for listings (código)
\lstdefinestyle{python}{
    language=Python,
    backgroundcolor=\color{gray!2},     % Color de fondo
    basicstyle=\ttfamily,               % Tipo y tamaño de fuente
    keywordstyle=\color{blue}\bfseries, % Color para palabras clave
    stringstyle=\color{miverde},        % Color para cadenas
    commentstyle=\color{red},           % Color para comentarios
    showspaces=false,                   % No mostrar espacios
    showstringspaces=false,             % No mostrar espacios en las cadenas
    frame=single,                       % Poner un marco alrededor del código
    breaklines=true,                    % Romper las líneas largas
    captionpos=b,                       % Posición del caption
    tabsize=4,                          % Tamaño de las tabulaciones
    escapeinside={\%*}{*)},             % Para incluir código LaTeX en los listings
    morekeywords={self}                 % Palabras clave adicionales
}
\lstset{basicstyle=\ttfamily}

%title
\title{Práctica 1} 

\author{Adrián Ferández Galán, César López Mantecón y Manuel Gómez-Plana Rodríguez}

\begin{document}

\begin{titlepage}
    \centering
   \includegraphics[width=0.9\textwidth]{uc3m.jpg} 
    {\Huge Universidad Carlos III\\
    
     \Large Ingeniería de la Ciberseguridad\\
     \vspace{0.5cm}
     Curso 2024-25}
    \vspace{2cm}

    {\Huge \textbf{Práctica 1} \par}
    \vspace{0.5cm}
    {\Large Extracción de contraseñas para binarios y permisos en ACL's\par}
    \vspace{8cm}

   \textbf{Ingeniería Informática, Cuarto curso}\\
    \vspace{0.2cm} 
    Adrián Fernández Galán       (NIA: 100472182, e-mail: 100472182@alumnos.uc3m.es)\\
    César López Mantecón         (NIA: 100472092, e-mail: 100472092@alumnos.uc3m.es)\\
    Manuel Gómez-Plana Rodríguez (NIA: 100472092, e-mail: 100472092@alumnos.uc3m.es)
    \vspace{0.5cm}

   
    \textbf{Prof .}Antonio Nappa\\
    
    \textbf{Grupo: } 81   
    
\end{titlepage}
\newpage

\renewcommand{\contentsname}{\centering Índice}
\tableofcontents

\newpage

\section{Introducción}
\label{sec:introduccion}
En este documento se recoge el proceso de desarrollo de la primera práctica de la asignatura \textit{Ingeniería de la Ciberseguridad}. En esta práctica hemos logrado obtener 9 \textit{flags} mediante el descubrimiento de contraseñas y ataque a archivos binarios. 

\section{Estrategias para el descubrimiento de contraseñas}
\label{sec:password}

Para extraer las contraseñas de los ejecutables se han empleado 2 estrategias distintas: uso de john the ripper para la obtención de contraseñas y ataque sobre los binarios.

\subsection{John The Ripper}
\label{sec:john}

\texttt{John} es una herramienta para la optención de contraseñas débiles a partir de su \textit{hash}. Dado el conocimiento que teníamos de las contraseñas hemos generado una \textit{wordlist} con todas las contraseñas posibles con el alfabeto proporcionado. Esto se reduce a las permutaciones de 5, 6 y 8 elementos de los conjuntos de caracteres usados para generar cada contraseña. 

Para generar la \textit{wordlist} se ha empleado el siguiente código de \textit{python}:

\lstset{style=python}
\begin{lstlisting}
import itertools

# charset level 1
charset_level1 = "abcdefg123456lab"

# Generate all permutations of length 5
permutations = itertools.permutations(charset_level1, 5)

with open("level1_wordlist.txt", "w+") as file:
    # Print the result
    for p in permutations:
        file.write(''.join(p) + "\n")
        
# charset level 2
charset_level2 = "abcdefg123456uc3m"

# Generate all permutations of length 6
permutations = itertools.permutations(charset_level2, 6)

with open("level2_wordlist.txt", "w+") as file:
    # Print the result
    for p in permutations:
        file.write(''.join(p) + "\n")

# charset level 3
charset_level3 = "abcdefg123456profe"

# Generate all permutations of length 8
permutations = itertools.permutations(charset_level3, 8)

with open("level3_wordlist.txt", "w+") as file:
    # Print the result
    for p in permutations:
        file.write(''.join(p) + "\n")

\end{lstlisting}

Sin embargo, no ha sido posible generar la \textit{wordlist} para \textit{level3} debido gran número de contraseñas posibles y, consecuentemente, al gran tamaño del archivo.

Usando este método, hemos podido extraer las contraseñas para los archivos \textit{level1} y \textit{level2}. En la siguiente tabla se recogen los tiempos que ha llevado obtener cada contraseña: 
\begin{table}
\begin{centering}
    \begin{tabular}{|c|c|c|}
        \hline
        Archivo & Longitud de la \textit{wordlist} en líneas & Tiempo\\
        \hline
        level1 & 524160 & 53s\\ 
        level2 & 8910720 & 383s\\
        level3 & 1764322560 & Not Finished\\
        \hline
    \end{tabular}
    \caption {\small Tiempos por contraseña usando \textit{john}}
\end{centering}
\end{table}

Para el archivo \textit{level3} no hemos empleado esta estategia debido al elevado tiempo de computación que precisa en comparación con la siguiente estrategia.

\subsection{Análisis de binarios}
\label{subsec:binary}
Hemos empleado 3 herramientas distintas para el análisis y la explotación de binarios: \texttt{objdump}, \texttt{Ghidra} y \texttt{gdb}. Mediante la primera, hemos analizado y comprendido el funcionamiento del ejecutable; logrando identificar dos subrutinas de gran importancia: \texttt{strcmp} y \texttt{decode\_password}. Esto lo hemos hecho a través del siguiente comando:

\begin{center}\texttt{objdump -D level1 level2 level3}\end{center}

Con esto hemos podido analizar el código ensamblador y descubrir que la función \texttt{decode\_password} recibe un \textit{string} y, a través de una serie de operaciones, la transforma in-situ.

Ejecutando los programas sobre un entorno controlado con \texttt{gdb} hemos detenido la ejecución al final de la subrutina \texttt{decode\_password} y leído el valor de la cadena de texto en memoria:

\begin{center}
    \begin{lstlisting}[caption=comandos para la obtención de contraseña en gdb]
gdb level1                  # ejecucion en entorno gdb
(gdb) break decode_password # breakpoint
(gdb) run                   # ejecucion hasta la rutina
(gdb) x/s $rdi              # contrasena cifrada
(gdb) finish                # salto a la instruccion ret
(gdb) x/s $rdi              # obtener contrasenia en claro
\end{lstlisting}
\end{center}

Dado que los tres binarios siguen la misma estrategia, ha sido posible obtener todas las contraseñas mediante este procedimiento.

Adicionalmente se ha decompilado el programa usando \texttt{Ghidra} con el fin de comprender mejor su funcionamiento. De esta forma hemos podido obtener el pseudocódigo en \texttt{C} de la función \texttt{decode\_password} y descubrir que la contraseña incluida en el binario está cifrada mediante el desplazamiento de 4 posiciones en la tabla ascii.

\subsection{Comparación de métodos}
\label{subsec:comparacion}

\section{ACL's}
\label{sec:acl}
Como se especifica en la práctica, la máquina virtual cuenta con 3 ficheros que el usuario \textit{rover} no tiene permiso para leer. Es por ello que, analizando la lista de ACL's de los siguientes ficheros, podemos ver quién tiene permisos para leer.
\section{Conclusión}
\label{sec:conclusion}

\end{document}
