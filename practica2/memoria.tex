\documentclass[]{article}
\usepackage{graphicx}
\usepackage[spanish]{babel}
\usepackage[a4paper, top=2.5cm, bottom=2.5cm, left=3cm, right=3cm]{geometry}
\usepackage[hidelinks]{hyperref}
\usepackage[T1]{fontenc}
\usepackage{listings}
\usepackage{xcolor}
\usepackage{float}

\definecolor{miverde}{rgb}{0,0.6,0}
\lstdefinelanguage{flag}{
  keywords={dbafcc, cafebb},
  keywordstyle=\bfseries,
  ndkeywords={},
  ndkeywordstyle=\color{miotrootroverde}\bfseries,
  identifierstyle={},
  sensitive=false,
  comment=[l]{//},
  morecomment=[s]{/*}{*/},
  commentstyle=\color{red}\ttfamily,
  stringstyle=\ttfamily,
  morestring=[b]',
  morestring=[b]"
}
% style for listings (código)
\lstdefinestyle{python}{
    language=Python,
    backgroundcolor=\color{gray!2},     % Color de fondo
    basicstyle=\ttfamily,               % Tipo y tamaño de fuente
    keywordstyle=\color{blue}\bfseries, % Color para palabras clave
    stringstyle=\color{miverde},        % Color para cadenas
    commentstyle=\color{red},           % Color para comentarios
    showspaces=false,                   % No mostrar espacios
    showstringspaces=false,             % No mostrar espacios en las cadenas
    frame=single,                       % Poner un marco alrededor del código
    breaklines=true,                    % Romper las líneas largas
    captionpos=b,                       % Posición del caption
    tabsize=4,                          % Tamaño de las tabulaciones
    escapeinside={\%*}{*)},             % Para incluir código LaTeX en los listings
    morekeywords={self}                 % Palabras clave adicionales
}

\lstdefinestyle{flag}{
    language=bash,
    backgroundcolor=\color{gray!2},     % Color de fondo
    basicstyle=\ttfamily,               % Tipo y tamaño de fuente
    keywordstyle=\bfseries, % Color para palabras clave
    stringstyle=\color{miverde},        % Color para cadenas
    commentstyle=\color{red},           % Color para comentarios
    showspaces=false,                   % No mostrar espacios
    showstringspaces=false,             % No mostrar espacios en las cadenas
    frame=single,                       % Poner un marco alrededor del código
    breaklines=true,                    % Romper las líneas largas
    captionpos=b,                       % Posición del caption
    tabsize=4,                          % Tamaño de las tabulaciones
    escapeinside={\%*}{*)},             % Para incluir código LaTeX en los listings
    morekeywords={self}                 % Palabras clave adicionales
}


\lstdefinestyle{bash}{
    language=bash,
    backgroundcolor=\color{gray!2},     % Color de fondo
    basicstyle=\ttfamily,               % Tipo y tamaño de fuente
    keywordstyle=\bfseries, % Color para palabras clave
    stringstyle=\color{miverde},        % Color para cadenas
    commentstyle=\color{red},           % Color para comentarios
    showspaces=false,                   % No mostrar espacios
    showstringspaces=false,             % No mostrar espacios en las cadenas
    frame=single,                       % Poner un marco alrededor del código
    breaklines=true,                    % Romper las líneas largas
    captionpos=b,                       % Posición del caption
    tabsize=4,                          % Tamaño de las tabulaciones
    escapeinside={\%*}{*)},             % Para incluir código LaTeX en los listings
    morekeywords={self}                 % Palabras clave adicionales
}
\lstset{basicstyle=\ttfamily}
\hypersetup{
    colorlinks=true, % Activa el color en los enlaces
    linkcolor=blue,  % Color de los enlaces internos (como secciones)
    urlcolor=blue,   % Color de los enlaces a URLs
}


%title
\title{Práctica 1} 

\author{Adrián Ferández Galán, César López Mantecón y Manuel Gómez-Plana Rodríguez}

\begin{document}

\begin{titlepage}
    \centering
   \includegraphics[width=0.9\textwidth]{uc3m.jpg} 
    {\Huge Universidad Carlos III\\
    
     \Large Ingeniería de la Ciberseguridad\\
     \vspace{0.5cm}
     Curso 2024-25}
    \vspace{2cm}

    {\Huge \textbf{Práctica 2} \par}
    \vspace{0.5cm}
    {\Large Intercepción de tráfico HTTP\par}
    \vspace{8cm}

   \textbf{Ingeniería Informática, Cuarto curso}\\
    \vspace{0.2cm} 
    Adrián Fernández Galán       (NIA: 100472182, e-mail: 100472182@alumnos.uc3m.es)\\
    César López Mantecón         (NIA: 100472092, e-mail: 100472092@alumnos.uc3m.es)\\
    Manuel Gómez-Plana Rodríguez (NIA: 100472310, e-mail: 100472310@alumnos.uc3m.es)
    \vspace{0.5cm}

   
    \textbf{Prof.} Antonio Nappa\\
    
    \textbf{Grupo: } 81   
    
\end{titlepage}
\newpage

%\renewcommand{\contentsname}{\centering Índice}
%\tableofcontents

\newpage

\lstset{style=bash}
\section{Introducción}
\label{sec:introduccion}
Este documento recoge el desarrollo de la segunda práctica de la asignatura
\textit{Ingeniería de la Ciberseguridad}. El objetivo de esta práctica es
desencriptar una serie de \textit{flags} contenidas en paquetes \texttt{http}
que atraviesan un \textit{proxy}.

Exiten dos clases de \textit{flags} contenidas en los paquetes:
\begin{itemize}
    \item \textbf{Tipo 1}: se trata de una cadena hexadecimal con un argumento
    llamado \textit{rotation}.
    \item \textbf{Tipo 2}: se trata de una cadena de caracteres con 3
    argumentos (\textit{Rotation, s} y \textit{op})
\end{itemize}

\section{Funcionamiento de \texttt{mitmproxy}}
\label{sec:proxy}

\texttt{Mitmproxy} es una herramienta para la consola de comandos que permite
el análisis interactivo y la modificaicón de tráfico \texttt{http}. Por
defecto, el proxy escucha sobre la dirección \textit{localhost} y puerto
\textit{8080}.

Adicionalmente, existe un proceso en constante ejecución llamado
\textit{down.sh} que hace una petición http sobre esta misma dirección.

\begin{lstlisting}[language=bash, caption=Resultado de ejecutar ps -e]
   PID TTY          TIME CMD
      1 pts/0    00:00:00 bash
    587 ?        00:00:00 sshd
  26023 ?        00:00:37 down.sh
 900981 ?        00:00:00 sshd
 900998 ?        00:00:00 sshd
 900999 pts/1    00:00:00 bash
 901270 ?        00:00:00 sleep
 901273 pts/1    00:00:00 ps
\end{lstlisting}

Se ha utilizado el comando \texttt{find} para localizar el archivo. Aprovechando el conocimiento previo que tenemos sobre la máquina, hemos empleado el usuario \textit{admin} para poder lanzar el comando sobre la raíz, ya que este usuario pertenece al grupo de superusuarios: 

\begin{lstlisting}[language=bash, caption=Resultado de ejecutar find]
sudo find / -type f -name down.sh
[sudo] password for admin:
./down.sh
\end{lstlisting}

Finalmente, analizando el contenido del archivo podemos ver el comando
utilizado para realizar la petición \texttt{http}.
\begin{lstlisting}[language=bash, caption=petición http a través del comando curl]
    /home/rover/curl-7.88.1/src/curl --insecure --proxy-insecure --proxy https://172.17.0.19:8080 https://172.17.0.4:4443/
\end{lstlisting}

Se observa que el comando redirige su petición a la ip y puerto donde escucha
\texttt{mitmproxy}. Adicionalmente, utiliza las flags \texttt{--insecure} y
\texttt{--proxy-insecure}, por lo que no verificará los certificados del proxy
y se permitirá la conexión. Con esto, la conexión se completa sin levantar
excepciones y podremos ser capaces de analizar las peticiones y respuestas a
través de la interfaz de \texttt{mitmproxy}.

\section{Descubrimiento de una flag tipo 1}
\label{sec:tipo1}

Las \textit{flags} están codificadas en hexadecimal y cuentan con un parámetro
llamado \textit{Rotation}. Este parámetro parece indicar que se ha cifrado a
través de una rotación en la cadena de texto. Al analizar varios paquetes se
aprecia que las flags de igual rotación cuentan con caracteres entre los
índices 10 y 15, lo que sugiere la utilización de una misma semilla para todas
las flags.

\lstset{style=flag}
\begin{lstlisting}[language=flag, caption=Varias flags para rotation 1 donde la semilla es dbafcc]
3193601502dbafcc86bdb57381081aec
3846408208dbafccc0399a91c4fbf1a3
4745131622dbafcc09340e7b0af6885f
\end{lstlisting}

Primeramente se trató de desencriptar las \textit{flags} rotando en ambos sentidos los valores hexadecimales de cada caracter. No obstante, esto no dio resultado. Adicionalmente, se probaron otras codificaciones como \texttt{ASCII}.

Posteriormente se observó que las flags con una diferencia de 6 rotaciones mantenían la misma semilla. Dado que la semilla está formada enteramente por letras y que la codificación hexadecimal cuenta con 6 caracteres alfabéticos, esto sugiere que el cifrado se ha aplicado por separado a letras y números.

\begin{lstlisting}[language=flag, escapechar=@, caption=Varias flag con la misma semilla y 6 rotaciones]
FLAG                              Rot
3193601502dbafcc86bdb57381081aec    1
3846408208dbafccc0399a91c4fbf1a3    1
4745131622dbafcc09340e7b0af6885f    1
9352846549dbafcc965695c25417c6d9    7
7173993205dbafccf2c4f4db398e6687    7
1047974726dbafcc05aacde26d253963    7
7512913585dbafcc9aeb0520366299c2    7
5483152252dbafcc26492c301ca7365e    7
2473284585dbafcc99892775f126bd30   13
0872242899dbafccfc622e537f7ba143   13
9735635169dbafcc24d5a92548bc1e09   13
7646156765dbafccd0beba631d833c5c   13
\end{lstlisting}

Finalmente, se ha hecho la prueba obteniendo una \textit{flag} válida a través
de un programa de python.

\lstset{style=python}
\begin{lstlisting}[language=Python, caption=Decodificación de una flag de rotación 1]
flag = "9443878358dbafcc223a8e670ba03537"
num_rotations = 1
letters_to_number = {'a': 0, 'b': 1, 'c': 2, 'd': 3, 'e': 4, 'f': 5}
number_to_letters = {0: 'a', 1: 'b', 2: 'c', 3: 'd', 4: 'e', 5: 'f'}
new_flag = ""

for i in range(len(flag)):
    c = flag[i]
    if c.isnumeric():
        new_c = (int(c) - num_rotations) % 10
    elif c in letters_to_number.keys():
        new_c = number_to_letters[(letters_to_number[c] - num_rotations) % 6]
    else:
        new_c = c
    new_flag += str(new_c)

print(new_flag)

\end{lstlisting}

El resultado obtenido ha sido el siguiente:
\texttt{8332767247cafebb112f7d569af92426}. Este resultado si ha sido aceptado
como correcto. La semilla empleada en la generación de las \textit{flags} es
\texttt{cafebb}.

A modo de curiosidad, dado que el cifrado de rotación es modular, la semilla se
mostrará en claro para una rotación igual al periodo. Es por esto que se puede
apreciar la semilla en las flags con rotación 6:

\begin{lstlisting}[language=flag, caption=Flags para rotacion 6]
3786566793cafebb39330ab797fbecfa
6350772345cafebb35f756efb7386a9d
0706541085cafebbeaf7963399a4d327
2847807583cafebbd1a054ded7579a71
1613995163cafebb528dd12a5f194769
\end{lstlisting}

\section{Descubrimiento de una flag tipo 2}
\label{sec:tipo2}

Las \textit{flags} de tipo 2 cuentan con un cifrado más complejo que requiere de varios pasos para su descrubrimiento. Los paquetes cuentan con 3 argumentos (\textit{Rotation, op} y \textit{s}) y una cadena de caracteres codificada en base 64.


\begin{lstlisting}[language=flag, caption=Ejemplo de flag tipo 2]
Hexadecimal Flag (Rotation: 15, s:1c7767, op:x): b'BlAEAAAGBVUHDlBTUgFNTQEEAwIAUwYHAQVVBlcDAFc='
\end{lstlisting}

Los argumentos se han interpretado de la siguiente forma:
\begin{itemize}
    \item \textbf{Rotation}: al igual que en las flags de tipo 1, se asume que indica la rotación en uno o varios diccionarios de caracteres.
    \item \textbf{s}: se interpreta como una semilla o contraseña usada en el cifrado.
    \item \textbf{op}: se interpreta como la operación usada en el cifrado.
    Dado que tiene un valor constante \texttt{x}, se asume que se trata de una operación \texttt{XOR} con la semilla, de forma similar a un \textit{cifrado Vernam}. 
\end{itemize}

Para descubrir la \textit{flag} se han seguido los pasos detallados a continuación.

\subsection{Exploración de la máquina y conversión a hexadecimal}

Al explorar la máquina se ha encontrado un directorio oculto en la raíz llamado \texttt{.ingsec2425} con un archivo \texttt{xor.c}. Este archivo cuenta con una función para aplicar una operación \texttt{XOR} sobre una cadena en hexadecimal y una contraseña. Con esta información, se ha codificado la \textit{flag} en hexadecimal y se ha aplicado esta función sobre ella y la semilla, obteniendo el siguiente resultado:

\lstset{style=bash}
\begin{lstlisting}[language=bash, caption=XOR sobre la flag]
    echo BlAEAAAGBVUHDlBTUgFNTQEEAwIAUwYHAQVVBlcDAFc= | base64 -d | xxd -p 
    > 0650040000060555070e505352014d4d01040302005306070105550657030057 # flag en hex

    /ingsec2425/xor lc7767 0650040000060555070e505352014d4d01040302005306070105550657030057
    > 37333337363134363039666463627a7a37333261376430303066623161343134
\end{lstlisting}

Es importante destacar que para este proceso hemos usado el usuario \textit{admin} para modificar los permisos del directorio y el ejecutable y poder lanzarlo desde el usuario \textit{rover}.

\subsection{Rotación}

Primeramente se probó a hacer la rotación descubierta en las \textit{flags} de tipo 1 sobre el resultado de la operación XOR. Esto no resultó en una \textit{flag} correcta.

Analizando más profundamente el resultado, la salida de la operación XOR cuenta con 64 caracteres, mientras que las flags de tipo 1 cuentan con 32 caracteres. Sabiendo que la cadena está codificada en hexadecimal, se transformó en \texttt{ASCII} para lograr una cadena con la longitud final esperada. Sobre la nueva cadena se volvió a aplicar la rotación, obteniendo el resultado final. Se ha usado el código de la parte anterior junto con el siguiente comando de linux para decodificar finalmente la flag:


\lstset{style=bash}
\begin{lstlisting}[language=bash, caption=Optención de la flag en claro]
    echo "37333337363134363039666463627a7a37333261376430303066623161343134" | xxd -r -p
    > 7337614609fdcbzz732a7d000fb1a414

    python3 decode_flag.py 7337614609fdcbzz732a7d000fb1a414
    > 6226503598ecbazz621f6c999ea0f303
\end{lstlisting}

\subsection{Pipeline de decodificación}

Se incluye este apartado a modo de resumen del proceso seguido en la decodificación de las \textit{flags} tipo 2.

\begin{enumerate}
    \item Traducción de la \textit{flag} de base 64 a hexadecimal.
    \item Operación XOR entre \textit{flag} y semilla.
    \item Codificación en \texttt{ASCII} del resultado anterior.
    \item Aplicación de la rotación descubierta para las \textit{flags} tipo 1.
\end{enumerate}

\section{Conclusión}
\label{sec:conclusion}
El análisis de tráfico de red es una de las principales actividades relacionadas con la ciberseguridad. La importancia de unas comunicaciones seguras a través de la red es vital no sólo para la protección de información crítica, sino también para el respecto de los derechos a la privacidad de los usuarios. Esta práctica nos ha permitido explorar el análisis e intercepción de tráfico web y familiarizarnos con herramientas como \texttt{mitmproxy}.

Trabajar con dos clases de \textit{flags} nos ha permitido experimentar como un cifrado algo más complejo multiplica en gran medida el esfuerzo necesario para la vulneración de la confidencialidad de un mensaje. Las flags de tipo 2 han requerido un esfuerzo mucho mayor para descubrir su descifrado. Al igual que vimos en cursos pasados, es importante que los algoritmos criptográficos reduzcan al máximo la información contenida en un mensaje cifrado para impedir el reconocimiento de patrones que le permitan al atacante descubrir su contenido. Sin embargo, en ambos tipos de \textit{flags} ha sido posible obtener información sobre el cifrado a través del análisis de distintas ocurrencias.

Para solucionar las vulnerabilidades existentes en el contexto de la práctica es necesario evitar el uso de marcadores como \texttt{--ssl-insecure} o \texttt{--proxy-insecure} que permiten la comunicación con proxys no autenticados. Adicionalmente, se sugiere usar algoritmos de cifrado probados y estándar como AES256.


\end{document}
