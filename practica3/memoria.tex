\documentclass[]{article}
\usepackage{graphicx}
\usepackage[spanish]{babel}
\usepackage[a4paper, top=2.5cm, bottom=2.5cm, left=3cm, right=3cm]{geometry}
\usepackage[hidelinks]{hyperref}
\usepackage[T1]{fontenc}
\usepackage{listings}
\usepackage{xcolor}
\usepackage{float}

\definecolor{miverde}{rgb}{0,0.6,0}
\lstdefinelanguage{flag}{
  keywords={dbafcc, cafebb},
  keywordstyle=\bfseries,
  ndkeywords={},
  ndkeywordstyle=\color{miotrootroverde}\bfseries,
  identifierstyle={},
  sensitive=false,
  comment=[l]{//},
  morecomment=[s]{/*}{*/},
  commentstyle=\color{red}\ttfamily,
  stringstyle=\ttfamily,
  morestring=[b]',
  morestring=[b]"
}
% style for listings (código)
\lstdefinestyle{python}{
    language=Python,
    backgroundcolor=\color{gray!2},     % Color de fondo
    basicstyle=\ttfamily,               % Tipo y tamaño de fuente
    keywordstyle=\color{blue}\bfseries, % Color para palabras clave
    stringstyle=\color{miverde},        % Color para cadenas
    commentstyle=\color{red},           % Color para comentarios
    showspaces=false,                   % No mostrar espacios
    showstringspaces=false,             % No mostrar espacios en las cadenas
    frame=single,                       % Poner un marco alrededor del código
    breaklines=true,                    % Romper las líneas largas
    captionpos=b,                       % Posición del caption
    tabsize=4,                          % Tamaño de las tabulaciones
    escapeinside={\%*}{*)},             % Para incluir código LaTeX en los listings
    morekeywords={self}                 % Palabras clave adicionales
}

\lstdefinestyle{flag}{
    language=bash,
    backgroundcolor=\color{gray!2},     % Color de fondo
    basicstyle=\ttfamily,               % Tipo y tamaño de fuente
    keywordstyle=\bfseries, % Color para palabras clave
    stringstyle=\color{miverde},        % Color para cadenas
    commentstyle=\color{red},           % Color para comentarios
    showspaces=false,                   % No mostrar espacios
    showstringspaces=false,             % No mostrar espacios en las cadenas
    frame=single,                       % Poner un marco alrededor del código
    breaklines=true,                    % Romper las líneas largas
    captionpos=b,                       % Posición del caption
    tabsize=4,                          % Tamaño de las tabulaciones
    escapeinside={\%*}{*)},             % Para incluir código LaTeX en los listings
    morekeywords={self}                 % Palabras clave adicionales
}


\lstdefinestyle{bash}{
    language=bash,
    backgroundcolor=\color{gray!2},     % Color de fondo
    basicstyle=\ttfamily,               % Tipo y tamaño de fuente
    keywordstyle=\bfseries, % Color para palabras clave
    stringstyle=\color{miverde},        % Color para cadenas
    commentstyle=\color{red},           % Color para comentarios
    showspaces=false,                   % No mostrar espacios
    showstringspaces=false,             % No mostrar espacios en las cadenas
    frame=single,                       % Poner un marco alrededor del código
    breaklines=true,                    % Romper las líneas largas
    captionpos=b,                       % Posición del caption
    tabsize=4,                          % Tamaño de las tabulaciones
    escapeinside={\%*}{*)},             % Para incluir código LaTeX en los listings
    morekeywords={self}                 % Palabras clave adicionales
}
\lstset{basicstyle=\ttfamily}
\hypersetup{
    colorlinks=true, % Activa el color en los enlaces
    linkcolor=blue,  % Color de los enlaces internos (como secciones)
    urlcolor=blue,   % Color de los enlaces a URLs
}


%title
\title{Práctica 3} 

\author{Adrián Ferández Galán, César López Mantecón y Manuel Gómez-Plana Rodríguez}

\begin{document}

\begin{titlepage}
    \centering
   \includegraphics[width=0.9\textwidth]{uc3m.jpg} 
    {\Huge Universidad Carlos III\\
    
     \Large Ingeniería de la Ciberseguridad\\
     \vspace{0.5cm}
     Curso 2024-25}
    \vspace{2cm}

    {\Huge \textbf{Práctica 3} \par}
    \vspace{0.5cm}
    {\Large Análisis de Malware\par}
    \vspace{8cm}

   \textbf{Ingeniería Informática, Cuarto curso}\\
    \vspace{0.2cm} 
    Adrián Fernández Galán       (NIA: 100472182, e-mail: 100472182@alumnos.uc3m.es)\\
    César López Mantecón         (NIA: 100472092, e-mail: 100472092@alumnos.uc3m.es)\\
    Manuel Gómez-Plana Rodríguez (NIA: 100472310, e-mail: 100472310@alumnos.uc3m.es)
    \vspace{0.5cm}

   
    \textbf{Prof.} Antonio Nappa\\
    
    \textbf{Grupo: } 81   
    
\end{titlepage}
\newpage

%\renewcommand{\contentsname}{\centering Índice}
%\tableofcontents

\newpage

\lstset{style=bash}
\section{Introducción}
\label{sec:introduccion}
Este documento recoge el desarrollo de la tercera práctica de la asignatura
\textit{Ingeniería de la Ciberseguridad}. El objetivo de esta práctica es el
análisis y comprensión de dos programas maliciosos para extraer una
\textit{flag} de cada uno. Para ello trabajaremos con una máquina virtual de
\href{https://www.kali.org/}{KaliLinux} totalmente aislada con el fin de poder
analizar los binarios tanto estáticamente como dinámicamente.

Los malwares son softwares con alguna intención maliciosa que puede poner a un
usuario, sus datos o un dispositivo en riesgo. Para esta práctica, será
fundamental identificar el comportamiento malicioso de cada uno de los
softwares y comprender su funcionamiento. Además de extraer las \textit{flags}
con el formato adecuado.

Adicionalmente, es importante tener en cuenta que contamos con información
acerca del autor de estos programas. Es por esto que centraremos gran parte de
nuestros esfuerzos en la identificación de cadenas donde un prefijo y un sufijo
se concatenen a una \textbf{semilla}, siendo esta típicamente una palabra
seguida de algún caracter repetido sobre la que se ha aplicado alguna clase de
cifrado de desplazamiento en un alfabeto.

\section{Parte 1}
\label{sec:type1}
El primer malware se trata de un programa que se ejecuta sobre un directorio.
Contamos con el código fuente, así como el binario ejecutable. Es importante
destacar que este último no cuenta con marcas de depuración, lo que dificulta
su análisis mediante decompiladores y entornos de \textit{debugging}.

Durante su análisis se emplearán las herramientas
\href{https://ghidra-sre.org/}{Ghidra} y
\href{https://man7.org/linux/man-pages/man1/gdb.1.html}{gdb}, además de otros
métodos propios.

\subsection{Análisis estático}
\label{subsec:analisis-estatico-1}

Durante este proceso se ha tratado de comprender del funcioanmiento del
programa, extraer el flujo habitual del código e identificar secciones o
funciones clave en su ejecución. Para ello se ha empleado \textit{Ghidra} 
para decompilar y comparar el pseudocódigo resultante con el código C y
verificar que ambos programas tienen el mismo comportamiento.

Finalmente se concluyó que el código l3.c y el ejecutable \texttt{runDir} son el mismo programa. Tras su análisis se extrajeron las siguientes conclusiones:
\begin{itemize}
    \item El programa realiza una llamada \texttt{fork} con el fin de evitar el
    uso de herramientas como \texttt{gdb} u otros entornos de debugging.
    \item El programa genera una flag de longitud pseudoaleatoria y de
    caracteres al azar mediante la concatenación de la semilla rotada
    ``srrqnn'' con un prefijo y un sufijo, también generados aleatoriamente.
    Esta flag se transforma mediante las funciones \texttt{transform\_flag} y
    \texttt{process\_buffer}.
    \item El programa procesa archivos con las extensiones .txt, .pdf, .jpg,
    .png y .doc; creando una copia y aplicando una transformación descrita en
    la función \texttt{process\_buffer}. A estos archivos transformados les
    inserta la \texttt{flag} transformada al final.
    \item Las funciones para transformar archivos o explorar directorios reciben una estructura donde se encuentra la \textit{flag} sin transformar y transformada.
\end{itemize}

Con todo lo anterior es fácil concluir que se trata de un programa que cifra u
ofusca archivos con extensiones concretas, creando una copia de seguridad
previamente. Destaca que trabaja siempre con copias de los archivos, nunca con
los originales. También, crea un archivo \textit{processed\_files.txt} donde
registran todos los archivos que se han procesado y se incluye la \textit{flag
transformada}.

\subsection{Análisis dinámico}
\label{subsec:analisis-dinámico-1}

El análisis dinámico ha permitido la extracción de las \textit{flags} transformada y sin transformar. Para ello ha sido necesario sortear la llamada a la función \texttt{fork} para que el padre procese el directorio, al igual que el hijo, y permitir su análisis con la herramienta \texttt{gdb}. Esto se ha logrado mediante el siguiente código:
\newpage
\begin{lstlisting}[caption=Instrucciones en gdb para esquivar la llamada fork]
(gdb) break fork@plt
(gdb) run
(gdb) finish
(gdb) set $rax=0
\end{lstlisting}

De esta forma el proceso padre recibirá un 0 como resultado del fork,
ejecutando el flujo habitual del proceso hijo. Con esto podemos extraer la
\textit{flag} a través de las siguientes instrucciones:

\begin{lstlisting}[caption=Instrucciones en gdb para extraer la flag]
(gdb) break process_directory
(gdb) finish
(gdb) x/s $rsi
(gdb) x/s $rsi + 64
\end{lstlisting}

Con la primera instrucción \texttt{x/s} podemos extraer la semilla antes de ser
transformada. Con la segunda, vemos el resultado de la transformación. Al
repetir esto varias veces, hemos comprobado que la \textit{flag} no cambia
entre ejecuciones, lo que ha levantado sospechas sobre la aleatoriedad de la
función \texttt{rand}. Al buscar información en el manual y realizar una prueba
en 3 máquinas distintas con diferentes sistemas operativos se ha confirmado que
la secuencia que genera esta función es siempre la misma, haciendo de la
generación de la \textit{flag} un proceso determinista. Para cualquier ejecución, la
cadena de caracteres tendrá una longitud de 26, donde el prefijo y sufijo serán
de longitud 10.

Complementariamente, se han analizado los ficheros generados por el ejecutable.
Esto ha permitido comprobar que, efectivamente, la flag escrita en el fichero
\textit{processed\_files.txt}, la extraída durante la ejecución y la flag
escrita al final de cada fichero es la misma. Se puede extraer la flag al final
del fichero a través del siguiente comando:

\begin{lstlisting}[caption=Obtención de la flag transformada concatenada al final del fichero]
tail -c 26 file.processed # 881531cc331308c2ca534a15c4
\end{lstlisting}

\subsection{Descripción del Malware}
\label{malware1}

El malware se trata de un cifrador de ficheros, lo que encaja con la
descripción de un \textit{ransomware}. Sin embargo, tiene ciertas
particularidades destacables. Lo primero, crea una copia de cada fichero en el
propio directorio, lo que no encaja del todo con el comportamiento típico de
este tipo de malware. También, no trabaja sobre el archivo original, cifrando
una copia. Esto se traduce en que, si una víctima ejecuta el binario en su
máquina, no existirían daños reales sobre sus datos u archivos. No obstante, si
se tiene en cuenta la naturaleza académica de este ejercicio, podemos ignorar
estas particularidades y reconocer el potencial daño que puede tener un
software de este tipo: la pérdida de acceso a ciertos ficheros en un directorio.

\subsection{Descubrimiento de la flag}
\label{flag1}

Para el descubrimiento de la \textit{flag} ha sido necesario rotar la semilla
descubierta. Dado que no cuenta con caracteres hexadecimales se ha empleado un
\textit{script} de \texttt{Python} para rotarla en el alfabeto inglés buscando
encontrar el patrón descrito en la introducción.

\begin{lstlisting}[caption=Resultado de rotar la semilla en el alfabeto inglés]
$ python rotate_seed.py 
0	srrqnn
1	tssroo
2	uttspp
3	vuutqq
4	wvvurr
5	xwwvss
6	yxxwtt
7	zyyxuu
8	azzyvv
9	baazww
10	cbbaxx
11	dccbyy
12	eddczz
13	feedaa
14	gffebb
15	hggfcc
16	ihhgdd
17	jiihee
18	kjjiff
19	lkkjgg
20	mllkhh
21	nmmlii
22	onnmjj
23	poonkk
24	qppoll
25	rqqpmm
\end{lstlisting}

De este resultado, la única semilla que cumple la condición de ser una palabra
seguida de dos letras es \textit{feedaa}. Adicionalmente cuenta exclusivamente
con caracteres hexadecimales. Lo que la convierte en
la única candidata viable para ser la semilla. Adicionalmente, se han
probado las rotaciones de ejercicios pasados sin resultado exitoso.

Finalmente, conociendo la longitud y ubicación de la semilla gracias al
análisis estático se ha conseguido entregar existosamente la semilla:
\texttt{6931FAC9DAFEEDAAB2B36C248B}. Esta semilla se ha generado replicando el
código de la función \texttt{generate\_flag}, eliminando la ofuscación y
empleando la semilla en claro.

\section{Parte 2}
\label{sec:type2}

\subsection{Análisis estático}
\label{subsec:analisis-estatico-2}

\subsection{Análisis dinámico}
\label{subsec:analisis-dinámico-2}

\subsection{Descripción del Malware}
\label{malware1}

\subsection{Descubrimiento de la flag}
\label{flag1}


\section{Conclusión}
\label{sec:conclusion}

\newpage
\section*{Anexo}

\subsection*{Código para la rotación de la flag en el alfabeto inglés}
\lstset{style=python}
\lstinputlisting[language=Python]{scripts/rotate_seed.py}

\end{document}
